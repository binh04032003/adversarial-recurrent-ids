%
% Beispieldokument für TU beamer theme
%
% v1.0: 12.10.2014
% 
\documentclass[xcolor={dvipsnames}]{beamer}
\usetheme{TU}

\title{SparseIDS: Learning Packet Sampling with Reinforcement Learning}

\author[Maximilian Bachl]{%
	Maximilian Bachl\email{maximilian.bachl@tuwien.ac.at} \and Fares Meghdouri\email{fares.meghdouri@tuwien.ac.at} \and Tanja Zseby\email{tanja.zseby@tuwien.ac.at} \and Joachim Fabini\email{joachim.fabini@tuwien.ac.at}
}

\institute{%
	Technische Universität Wien, Vienna, Austria
}

%\session{XXXX \#}

%Kann angepasst werden, wie es beliebt. Entweder das Datum der Präsentation oder das Datum der aktuellen Präsentations-Version.
%\date[\the\day.\the\month.\the\year]{\today}
\date[July 1, 2020]{July 1, 2020}

\begin{document}

\maketitle

\section{Introduction}

\begin{frame}{Recurrent Neural Networks (RNNs)}
\begin{itemize}
\item RNNs are neural networks for \textbf{sequences}
\item Good performance for network traffic (\textit{Explainability and Adversarial Robustness for RNNs}, Hartl, Bachl, Fabini, Zseby, 2020) with \textbf{supervised} training
\item Can detect intrusion \textbf{before} attack is over
\end{itemize}
\end{frame}

\begin{frame}{Sampling for RNNs}
\begin{itemize}
\item Processing every packet by a neural network is \textbf{resource-intensive}
\item Often, attack is apparent after the first packets of a flow $\rightarrow$ not \textbf{necessary} to continue inspecting packets
\item Some packets might contain \textbf{no information} that is useful for classification
\end{itemize}

\pause
\begin{block}{Insight:}
Only process packets if the are useful for the classifier! 
\end{block}

\end{frame}

\section{Concept}

\begin{frame}{Defining the goal}
Classifier should
\begin{itemize}
\item be \textbf{reasonably accurate}
\item \textbf{skip} packets if they \textbf{don't provide substantial benefit}
\end{itemize}

\pause
\begin{block}{Insight:}
Specify tradeoff between accuracy/sparsity 
\end{block}

\end{frame}

\begin{frame}{Defining the goal}
Optimize
\begin{align*}
\text{Objective} = \text{Accuracy} + \alpha \times \text{Sparsity}
\end{align*}
\end{frame}

\begin{frame}{Neural Network architecture}

\centering
\includegraphics[width=0.6\columnwidth]{{neural_network}.pdf}

\end{frame}

% -------------
% Last page
% -------------
\makelastslide

\end{document}